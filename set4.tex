\documentclass[12pt]{article}
\usepackage{amssymb}
\usepackage{amsthm}
\usepackage{amsmath}
\usepackage{float}
\usepackage{graphicx}
\usepackage{breqn}
\usepackage[margin=1in]{geometry}
\newcommand{\parder}[2]{\frac{\partial}{\partial{#2}}#1}
\newcommand{\parderpow}[3]{\frac{\partial^{#3}}{\partial{#2}^{#3}}#1}
\newcommand{\parderplace}[2]{\frac{\partial{#1}}{\partial{#2}}}
\newcommand{\parderpowplace}[3]{\frac{\partial^{#3}{#1}}{\partial{#2}^{#3}}}
\newcommand{\Exp}{\mathrm{Exp}}
\newcommand{\bra}[1]{\langle #1}
\newcommand{\ket}[1]{\lvert #1 \rangle}
\newcommand{\highlight}[2]{\colorbox{#2}{$\displaystyle #1$}}
\renewcommand{\thesubsection}{(\alph{subsection})}

\usepackage{minted}

\begin{document}
\title{Ph 121a Assignment 4: Molecular Dynamics}
\author{Nicholas Meyer}

\maketitle

\section{Approach}

Here, a neutral noble gas was modeled using the Lennard-Jones potential

\begin{align}
  U_{LJ}(r) = \epsilon \left[\left(\frac{r_{\text{min}}}{r}\right)^{12}
  - 2\left(\frac{r_{\text{min}}}{r}\right)^6\right]
\end{align}

\section{Implementation}

The routines for calculating the forces between particles and updating the
positions through the Langevin-Verlet integrator are in C++.
Control of the running simulations and plotting are done in python, using
\mintinline{python}/ctypes/ to interface with the C++ code.

\subsection{Use of double precision instead of single floats}

When single precision floating point numbers were used for particle positions
and simulation parameters, particles would accelerate toward the bottom left
corner of the screen. The drift was most noticeable for temperatures close to
zero. This was likely due to the accumulation of rounding error (which was
for some reason consistently biased toward rounding down) over many
iterations.

This behavior was fixed by switching to double precision floats.

\section{Running the suite}

First, it is necessary to build the dynamic library by running

\begin{minted}{bash}
make molec.so
\end{minted}

Then, the visualizer can be run from the command line, i.e.

\begin{minted}{bash}
python visualizer.py 1.0 movie 1000
\end{minted}

to make a movie with 1000 frames of the system starting at a solid and
evolving when the temperature is suddenly set to one, or

\begin{minted}{bash}
python visualizer.py 5.0
\end{minted}

to view the animation for $T = 5$.

\section{Plots}

\subsection{Solid}

To view as an animation, run

\begin{minted}{bash}
python 
\end{minted}

\subsection{Liquid}

To view as an animation, run

\begin{minted}{bash}
python
\end{minted}

\subsection{Gas}

To view as an animation, run

\begin{minted}{bash}
python 
\end{minted}

\section*{Appendix}

\subsection*{Source Code}

\subsubsection*{molec.hpp}

\inputminted{c++}{molec.hpp}

\newpage

\subsubsection*{Molec.py}

\inputminted{python}{Molec.py}

\newpage

\subsubsection*{visualize.py}

\end{document}
